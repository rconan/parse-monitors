\documentclass{gmto}

\usepackage{booktabs}
\usepackage{longtable}

\DocID{GMT-DOC-05222}
\DocVersion{1.0}
\DocStatus{Draft}

\addbibresource{mount.bib}

\title{CFD Thermal Boundary Updates}
\author{R. Conan, K. Vogiatzis, H. Fitzpatrick}
\date{\today}

\begin{document}
\maketitle
\tableofcontents
\listoffigures
\listoftables

\clearpage

\section{Purpose}
\label{sec:purpose}

The analysis reported in the document investigates the effects on dome seeing of changing the temperature of some surfaces in the CFD baseline from 1 degree above ambient to 2 degrees below ambient.

\section{Introduction}
\label{sec:introduction}

The GMT CFD Baseline Model is described in \cite{GMT.DOC.05211}, it includes the list and specifications of all the thermal boundary conditions (TBCs).
A complete overview of the results from the GMT CFD Baseline simulations is available in \cite{GMT.DOC.05214}.
The updated TBCs are given in the \emph{Surface Temp Reqt Update} accompanying document.
Four new cases, identical to the GMT CFD Baseline, but with the updated TBCs have been run and the V and H band PSSn were derived from ray tracing through the volume of the random field of refraction index.

The following table gives, for each case, the WFE RMS and the V and H band PSSn for the updated and default TBCs.
The WFE RMS is given as the mean over 400s and the V and H band PSSn are derived from a 400s long exposure dome seeing image.
\begin{longtable}{*{4}{c}|*{3}{r}|*{3}{r}}\toprule
 \multicolumn{4}{c|}{\textbf{CFD Cases}} & \multicolumn{3}{|c|}{\textbf{Updated TBC}} & \multicolumn{3}{|c}{\textbf{Default TBC}} \\\midrule
  Zen. & Azi. & Cfg. & Wind & WFE & PSSn & PSSn & WFE & PSSn & PSSn \\
  - & -    & -    &  -   & RMS & -  & - & RMS & - & -  \\
  $[deg]$  & $[deg.]$ & - & $[m/s]$ & $[nm]$& V & H & $[nm]$ & V & H \\\hline
   30 &  45 & cd &  7 &    730 & 0.9184 & 0.9091 &    785 & 0.9101 & 0.8978 \\
  30 &  45 & os &  7 &    755 & 0.9003 & 0.8946 &    776 & 0.8863 & 0.8530 \\
  30 & 135 & cd &  7 &   1712 & 0.7959 & 0.7431 &   1544 & 0.7921 & 0.7299 \\
  30 & 135 & os &  7 &   1097 & 0.9365 & 0.9282 &   1232 & 0.9131 & 0.8966 \\
\bottomrule
\end{longtable}

In the following,
each section includes a snapshot of the gradient of the refraction index for the whole Observatory (for both the updated and default TBCs) and time series plots of the WFE RMS and V and H PSSn.
The dome seeing related data is sampled at 5Hz.
There are PSSn plots for both short exposure (SE) and long exposure (LE) PSSn.
SE PSSn have the same sampling than the dome seeing data (5Hz).
The LE PSSn values are derived from cummulative dome seeing images sampled at 1Hz (for the LE PSSn, the meaning of the x-axis is exposure time).


\clearpage
\section{30 zenith - 45 azimuth - Closed vents/Deployed wind screen - 7m/s}
\subsection{Gradient of index of refraction}
\paragraph{Updated TBC}\mbox{}\\
\includegraphics[width=0.8\textwidth]{{{"/fsx/Baseline2021/Baseline2021/Baseline2021/THBOUND2/zen30az045_CD7/scenes/RI_tel_RI_tel_7.000000e+02"}}}
\paragraph{Default TBC}\mbox{}\\
\includegraphics[width=0.8\textwidth]{{{"/fsx/Baseline2021/Baseline2021/Baseline2021/CASES/zen30az045_CD7/scenes/RI_tel_RI_tel_7.000000e+02"}}}

\subsection{Wavefront error RMS}
\includegraphics[width=0.7\textwidth]{{{"/fsx/Baseline2021/Baseline2021/Baseline2021/THBOUND2/zen30az045_CD7/dome-seeing_wfe-rms"}}}
\clearpage
\subsection{PSSn}
\subsubsection{V}
\includegraphics[width=0.7\textwidth]{{{"/fsx/Baseline2021/Baseline2021/Baseline2021/THBOUND2/zen30az045_CD7/dome-seeing_v-pssn"}}}
\subsubsection{H}
\includegraphics[width=0.7\textwidth]{{{"/fsx/Baseline2021/Baseline2021/Baseline2021/THBOUND2/zen30az045_CD7/dome-seeing_h-pssn"}}}


\clearpage
\section{30 zenith - 45 azimuth - Open vents/Stowed wind screen - 7m/s}

\subsection{Gradient of index of refraction}
\paragraph{Updated TBC}\mbox{}\\
\includegraphics[width=0.8\textwidth]{{{"/fsx/Baseline2021/Baseline2021/Baseline2021/THBOUND2/zen30az045_OS7/scenes/RI_tel_RI_tel_7.000000e+02"}}}
\paragraph{Default TBC}\mbox{}\\
\includegraphics[width=0.8\textwidth]{{{"/fsx/Baseline2021/Baseline2021/Baseline2021/CASES/zen30az045_OS7/scenes/RI_tel_RI_tel_7.000000e+02"}}}

\subsection{Wavefront error RMS}
\includegraphics[width=0.7\textwidth]{{{"/fsx/Baseline2021/Baseline2021/Baseline2021/THBOUND2/zen30az045_OS7/dome-seeing_wfe-rms"}}}
\clearpage
\subsection{PSSn}
\subsubsection{V}
\includegraphics[width=0.7\textwidth]{{{"/fsx/Baseline2021/Baseline2021/Baseline2021/THBOUND2/zen30az045_OS7/dome-seeing_v-pssn"}}}
\subsubsection{H}
\includegraphics[width=0.7\textwidth]{{{"/fsx/Baseline2021/Baseline2021/Baseline2021/THBOUND2/zen30az045_OS7/dome-seeing_h-pssn"}}}


\clearpage
\section{30 zenith - 135 azimuth - Closed vents/Deployed wind screen - 7m/s}

\subsection{Gradient of index of refraction}
\paragraph{Updated TBC}\mbox{}\\
\includegraphics[width=0.8\textwidth]{{{"/fsx/Baseline2021/Baseline2021/Baseline2021/THBOUND2/zen30az135_CD7/scenes/RI_tel_RI_tel_7.000000e+02"}}}
\paragraph{Default TBC}\mbox{}\\
\includegraphics[width=0.8\textwidth]{{{"/fsx/Baseline2021/Baseline2021/Baseline2021/CASES/zen30az135_CD7/scenes/RI_tel_RI_tel_7.000000e+02"}}}

\subsection{Wavefront error RMS}
\includegraphics[width=0.7\textwidth]{{{"/fsx/Baseline2021/Baseline2021/Baseline2021/THBOUND2/zen30az135_CD7/dome-seeing_wfe-rms"}}}
\clearpage
\subsection{PSSn}
\subsubsection{V}
\includegraphics[width=0.7\textwidth]{{{"/fsx/Baseline2021/Baseline2021/Baseline2021/THBOUND2/zen30az135_CD7/dome-seeing_v-pssn"}}}
\subsubsection{H}
\includegraphics[width=0.7\textwidth]{{{"/fsx/Baseline2021/Baseline2021/Baseline2021/THBOUND2/zen30az135_CD7/dome-seeing_h-pssn"}}}


\clearpage
\section{30 zenith - 135 azimuth - Open vents/Stowed wind screen - 7m/s}

\subsection{Gradient of index of refraction}
\paragraph{Updated TBC}\mbox{}\\
\includegraphics[width=0.8\textwidth]{{{"/fsx/Baseline2021/Baseline2021/Baseline2021/THBOUND2/zen30az135_OS7/scenes/RI_tel_RI_tel_5.900000e+02"}}}
\paragraph{Default TBC}\mbox{}\\
\includegraphics[width=0.8\textwidth]{{{"/fsx/Baseline2021/Baseline2021/Baseline2021/CASES/zen30az135_OS7/scenes/RI_tel_RI_tel_5.900000e+02"}}}

\subsection{Wavefront error RMS}
\includegraphics[width=0.7\textwidth]{{{"/fsx/Baseline2021/Baseline2021/Baseline2021/THBOUND2/zen30az135_OS7/dome-seeing_wfe-rms"}}}
\clearpage
\subsection{PSSn}
\subsubsection{V}
\includegraphics[width=0.7\textwidth]{{{"/fsx/Baseline2021/Baseline2021/Baseline2021/THBOUND2/zen30az135_OS7/dome-seeing_v-pssn"}}}
\subsubsection{H}
\includegraphics[width=0.7\textwidth]{{{"/fsx/Baseline2021/Baseline2021/Baseline2021/THBOUND2/zen30az135_OS7/dome-seeing_h-pssn"}}}



\printbibliography

\end{document}
