\documentclass{gmto-book}

\DocID{GMT-DOC-05213}
\DocVersion{1.0}
\DocStatus{Draft}

\usepackage{dingbat}
\usepackage{booktabs}
\usepackage{longtable}

\addbibresource{mount.bib}

\setcounter{tocdepth}{5}

\title{GMT Computational Fluid Dynamics Mount Wind Loads}
%\subtitle{Mount Wind Loads}
\author{R. Conan, K. Vogiatzis, H. Fitzpatrick}
\date{\today}

\begin{document}

\maketitle

\clearpage

\section*{Signatures}
\vspace{1cm}
\subsection*{Author}
\vspace{1.5cm}
%\tabulinesep=1em
\begin{tabu} to \linewidth {X[3,l]X[1,l]}
  \rule{\linewidth}{.1pt} & \rule{\linewidth}{.1pt} \\
  Name, title & Date
\end{tabu}
\vspace{1.5cm}
\subsection*{Approvers}
\vspace{1.5cm}
%\tabulinesep=1em
\begin{tabu} to \linewidth {X[3,l]X[1,l]}
  \rule{\linewidth}{.1pt} & \rule{\linewidth}{.1pt} \\
  Name, title & Date \\[1cm]
  \rule{\linewidth}{.1pt} & \rule{\linewidth}{.1pt} \\
  Name, title & Date
\end{tabu}

\clearpage

\section*{Revision Log}

\begin{revisions}
  1.0 & 11.17.21 & All & None & Initial version & R. Conan \\  
  1.0 & 02.24.22 & All & None & Add mount wind loads cases definition and M1 segments net forces and moments & R. Conan \\  
  1.0 & 03.10.22 & All & None & Add optical metrics performance table & R. Conan \\  
  1.0 & \today & All & None & Fix PMT nodes  & R. Conan \\  
\end{revisions}

\clearpage

\tableofcontents
\listoffigures
\listoftables

\clearpage

\chapter{Introduction}
\label{sec:introduction}

\section{Purpose}
\label{sec:purpose}

This document describes the wind loads that the GMTO Mount Element is providing
to the mount vendor.

\section{Introduction}

Following the mount PDR\cite{MOUNT-PDR}, the CFD telescope model has been updated to take
into account the numerous design changes.
The updated CFD model is described in GMT-DOC--05211\cite{GMT.DOC.05211}.

The mount wind loads are detailed in  chapter~\ref{cfd-wind-loads}.
All the different cases correspond to the telescope pointing at a 30 degree zenith angle.
The telescope azimuth is given with respect to the 30$^o$-NNE predominant wind
direction at the GMT site.
The enclosure configuration is either vents open and wind screen stowed (OS) for the
2m/s and 7m/s cases or vents closed and wind screen deployed (CD) for the 12m/s
cases.
Table~\ref{tab:cases} identifies the CFD cases for the mount wind loads.
\begin{table}
  \centering
  \begin{tabular}{c|ccc}\toprule
    Azimuth [degree]  & 2m/s (OS) & 7m/s (OS) & 12m/s (CD) \\\midrule
      0               & \checkmark& \checkmark& \checkmark \\
     45               & \checkmark& \checkmark& \checkmark \\
     90               & \checkmark& \checkmark& \checkmark \\
    135               & & \checkmark&  \\
    180               & & & \checkmark \\\bottomrule
  \end{tabular}
  \caption{CFD settings for the mount wind loads cases.}
  \label{tab:cases}
\end{table}
The definition of the wind load cases according to the mount nomenclature can be found in Table~\ref{tab:mount-cases}.
\begin{table}
  \centering
  \begin{tabular}{cccc}\toprule
    & Wind speed & Azimuth & Enclosure\\
    & [m/s]       & [deg]   & - \\\midrule
    WLC0023 &  2  &  45     & OS \\
    WLC0024 &  2  &  90     & OS \\
    WLC0025 &  2  &   0     & OS \\
    WLC0026 &  7  &  45     & OS \\
    WLC0027 &  7  &  90     & OS \\
    WLC0028 &  7  & 135     & OS \\
    WLC0029 & 12  &   0     & CD \\
    WLC0030 & 12  &  90     & CD \\
    WLC0031 &  7  &   0     & OS \\
    WLC0032 & 12  & 180     & CD \\
    WLC0033 & 12  &   0     & CD \\
    WLC0034 & 12  &  45     & CD \\
    \bottomrule
  \end{tabular}
  \caption{CFD cases definition according to the mount wind loads nomenclature.}
  \label{tab:mount-cases}
\end{table}



%% Wind load as a function of wind speed:

%% \begin{tabular}{c|ccc}\toprule
%%  Wind [m/s] & 0 deg. & 45 deg. & 90 deg. \\\midrule
%%   \multicolumn{4}{c}{Left shutter} \\\midrule
%%      2  & 233  & 366 & 450 \\
%%      7   & 212  & 362 & 435 \\
%%     12   & 261  & 406 & 555 \\\midrule
%%   \multicolumn{4}{c}{Right shutter} \\\midrule
%%      2   & 226  & 305 & 182 \\
%%      7   & 221  & 336 & 175 \\
%%     12   & 261  & 416 & 191 \\\bottomrule
%%   \end{tabular}
 
%%% Local Variables:
%%% mode: latex
%%% TeX-master: "gmto.cfd2021.mount"
%%% End:


%\input{mount.chapter}
\chapter{Wind loads}
\label{cfd-wind-loads}

The wind loads on the telescope have been gathered in 8 groups:
\begin{itemize}
\item C-Rings
\item GIR
\item LGS
\item M1 assembly: the M1 7 segments and the M1 cell
\item M1 covers: the 6 outer and inner deployable segment covers
\item M2 \& Top-End
\item Trusses: inluding the cable truss
\item Platforms \& Trays
\end{itemize}

The simulations take time to converge to a stationary solution and this convergence time is a function of the initial conditions but strongly depends on the wind speed and the enclosure configuration.
The duration of the simulations have been set to 1200s, 700s and 900s for the 2m/s, 7m/s and 12m/s CFD runs, respectively.
It is usually the case that the last 400s of the simulations are stationary and
the statistical quantity reported in the forces and moments table (Table~\ref{tab:windloads}) have been derived from
the last 400s of the corresponding time series.

The chapter is divided into 11 sections, one section for each CFD case.
Each case is defined according to a telescope zenith and azimuth angles, a wind
speed and an enclosure configuration as defined in Table~\ref{tab:cases} and Table~\ref{tab:mount-cases}.

Each section in  chapter~\ref{cfd-wind-loads} includes time series plots of the total forces and total moments and the power spectrum densities of the total forces and total moments.
Both, forces and moments are sampled at 20Hz.
Forces and moments vectors are given with respect to the OSS coordinates system. 

The wind loads have been applied to the detailed Integrated Modeling FEM with
the mount controller turned on.
At each time step, the rigid body motions of M1 and M2 segments and the node
displacements at the PMT interfaces are recorded.

The rigid body motions are converted into wavefront error subject to the LTAO
rejection transfer functions.
From the wavefront, the focal plane tip-tilt and segment piston are derived.
The results are reported in Table~\ref{im-loads}.

The PMT transformation matrices (PMT1 and PMT2) are applied to the PMT nodes and
the results are filtered with the LTAO rejection transfer functions.
The resulting segment tip-tilts and segment pistons are further converted into
WFE RMS.
The results are reported in Table~\ref{mount-pmts}.


\begin{longtable}{cccr||rrrrr}
  %\centering
  %\begin{tabular}{cccr|rrr}
  \caption{Optical performance under wind load disturbances for the LTAO
 Observing Performance Mode.} \label{tab:long} \\\hline\hline
 % \multicolumn{4}{c|}{-} & \multicolumn{3}{|c}{\textbf{Natural Seeing Observing
 %     Performance Mode}} \\
 % \multicolumn{4}{c|}{\textbf{CFD cases}}  &
  \multicolumn{7}{|c}{\textbf{LTAO OPM}}
  \\\hline\hline
  Zen. & Azi. & Cfg. & Wind  & WFE & TT & Piston \\
  - & -    & -    &  -   &  RMS & RSS  & RMS  \\
  $[deg]$  & $[deg.]$ & - & $[m/s]$ & $[nm]$& $[mas]$&
  $[nm]$ \\\hline\hline
  \endfirsthead

  Zen. & Azi. & Cfg.  & WFE & TT & Piston \\
  - & -    & -    &  -   &  RMS & RSS  & RMS  \\
  $[deg]$  & $[deg.]$ & - & $[m/s]$ & $[nm]$& $[mas]$&
  $[nm]$ \\\hline\hline
  
  \endhead

%  \multicolumn{7}{r}{{\emph{Continued on next page}}} \\ 
  \endfoot
  \hline \hline
  \endlastfoot
  30 & 0 & o.s. & 2    &     7 & 0 &    9 \\\hline  
  30 & 45 & o.s. & 2   &     5 & 0 &    7 \\\hline  
  30 & 90 & o.s. & 2   &     5 & 0 &    7 \\\hline\hline  
  30 & 0 & o.s. & 7    &   122 & 1 &  167 \\\hline  
  30 & 45 & o.s. & 7   &    61 & 1 &   85 \\\hline  
  30 & 90 & o.s. & 7   &    93 & 1 &  126 \\\hline  
  30 & 135 & o.s. & 7  &    91 & 0 &  130 \\\hline\hline  
  30 & 0 & c.d. & 12   &   132 & 1 &  176 \\\hline  
  30 & 45 & c.d. & 12  &    82 & 0 &  100 \\\hline  
  30 & 90 & c.d. & 12  &    99 & 0 &  107 \\\hline  
  30 & 180 & c.d. & 12 &   240 & 0 &  342 \\  
  
  %\end{tabular}
  %\caption{Wind loadind optical performance}
  \label{im-loads}
\end{longtable}

\begin{longtable}{cccr||rrrrr}
  %\centering
  %\begin{tabular}{cccr|rrr}
  \caption{Mount PMTs.} \label{tab:long} \\\hline\hline
 % \multicolumn{4}{c|}{-} & \multicolumn{3}{|c}{\textbf{Natural Seeing Observing
 %     Performance Mode}} \\
 % \multicolumn{4}{c|}{\textbf{CFD cases}}  &
  \multicolumn{7}{|c}{\textbf{LTAO OPM}}
  \\\hline\hline
  Zen. & Azi. & Cfg. & Wind  & WFE & TT & Piston \\
  - & -    & -    &  -   &  RMS & RSS  & RMS  \\
  $[deg]$  & $[deg.]$ & - & $[m/s]$ & $[nm]$& $[nm]$&
  $[nm]$ \\\hline\hline
  \endfirsthead

  Zen. & Azi. & Cfg.  & WFE & TT & Piston \\
  - & -    & -    &  -   &  RMS & RMS  & RMS  \\
  $[deg]$  & $[deg.]$ & - & $[m/s]$ & $[nm]$& $[nm]$&
  $[nm]$ \\\hline\hline
  
  \endhead

%  \multicolumn{7}{r}{{\emph{Continued on next page}}} \\ 
  \endfoot
  \hline \hline
  \endlastfoot
  30 & 0 & o.s. & 2    &   7 &  0 &   7 \\\hline  
  30 & 45 & o.s. & 2   &   8 &  0 &   8 \\\hline  
  30 & 90 & o.s. & 2   &   8 &  0 &   8 \\\hline\hline  
  30 & 0 & o.s. & 7    &  92 &  8 &  92  \\\hline  
  30 & 45 & o.s. & 7   &  74 &  3 &  74 \\\hline  
  30 & 90 & o.s. & 7   &  83 &  4 &  83 \\\hline  
  30 & 135 & o.s. & 7  &  92 &  4 &  92 \\\hline\hline  
  30 & 0 & c.d. & 12   & 233 & 22 & 232  \\\hline  
  30 & 45 & c.d. & 12  & 280 & 10 & 280  \\\hline  
  30 & 90 & c.d. & 12  & 361 & 14 & 361  \\\hline  
  30 & 180 & c.d. & 12 & 365 & 10 & 365  \\  
  
  %\end{tabular}
  %\caption{Wind loadind optical performance}
  \label{mount-pmts}
\end{longtable}


\input{mount.groups.table}
\input{mount.time-series}

\appendix

%\chapter{50s total forces time series}
%\input{mount.50s-plots}

\chapter{Group PSDs}
\input{mount.psds.appendix}

\chapter{Wind load groups}
\input{mount.groups}


\printbibliography

\end{document}

%%% Local Variables:
%%% mode: latex
%%% TeX-master: "gmto.cfd2021.mount"
%%% End:
