\section{Purpose}
\label{sec:purpose}

This document describes the results from the 2021 CFD baseline simulations.

\section{Introduction}

Following the mount PDR\cite{MOUNT-PDR}, the CFD telescope model has been updated to take
into account the numerous design changes.
The updated CFD model is described in GMT-DOC--05211\cite{GMT.DOC.05211}.

There is a total of 77 CFD cases covering 3 telescope pointing angle (wrt. to zenith): 0$^o$, 30$^o$ and 60$^o$, 5 telescope azimuth angles wrt. the 30$^o$-NNE predominant wind direction at the GMT site: 0$^o$, 45$^o$, 90$^o$, 135$^o$ and 180$^o$, 5 wind speeds: 2m/s, 7m/s, 12m/s, 17m/s and 22m/s and 3 enclosure configurations: open vents/stowed wind screen (OS), closed vents/deployed wind screen (CD) and closed vents/stowed wind screen (CS).
Table~\ref{tab:cases} identifies all the CFD cases.
\begin{table}
  \centering
  \begin{tabular}{c|ccccc}\toprule
    Azimuth [degree] & \multicolumn{5}{c}{Wind speed [m/s]} \\\hline
     - & 2 & 7  & 12 & 17 & 22 \\\midrule
    \multicolumn{6}{c}{0$^o$ zenith}\\\hline
      0               & OS   & OS/CD & CD    & CD    & -    \\
     45               & OS   & OS/CD & CD    & CD    & -    \\
     90               & OS   & OS/CD & CD    & CD$^!$    & -    \\
    135               & OS   & OS/CD & CD    & CD    & -    \\
    180               & OS   & OS/CD$^!$ & CD    & CD    & -    \\\hline
    \multicolumn{6}{c}{30$^o$ zenith}\\\hline
      0               & OS   & OS/CD & CD    & CD    & CD    \\
     45               & OS   & OS/CD & CD    & CD    & CD    \\
     90               & OS   & OS/CD & CD    & CD    &     \\
    135               & OS   & OS/CD & CD    & CD    &     \\
    180               & OS   & OS/CD & CD    & CD    &     \\\hline
    \multicolumn{6}{c}{60$^o$ zenith}\\\hline
      0               & OS$^!$   & OS/CS & CS    & CS    & -    \\
     45               & OS   & OS/CS & CS    & CS    & -    \\
     90               & OS   & OS/CS & CS    & CS    & -    \\
    135               & OS   & OS/CS & CS    & CS    & -    \\
    180               & OS   & OS/CS & CS    & CS    & -    \\
    \bottomrule
  \end{tabular}
  \caption[CFD GMT enclosure configuration]{CFD GMT enclosure configuration (OS: open vents/stowed wind screen, CD: closed vents/deployed wind screen, CS: closed vents/stowed wind screen) for all the CFD cases. ($^!$ indicates cases that have failed and that not presented in this report.)}
  \label{tab:cases}
\end{table}

The report is divided into 3 parts, each corresponding to a specific data products from the CFD simulations: dome seeing in part~\ref{dome-seeing}, wind loads in part~\ref{wind-loads} and heat transfer coefficients (HTC) in part~\ref{htc}.
All the parts are divided into 3 chapters, one chapter for each telescope zenith angle.
In each chapter, there are as many sections as CFD cases at that particular zenith angle.

\paragraph{Dome seeing}

Dome seeing results are given in part~\ref{dome-seeing}.
Each chapter starts with a table with the WFE RMS and V and H PSSn values for both the current (2021) and the former (2020) set of simulations.
Each section includes a snapshot of the gradient of the refraction index for the whole Observatory and time series plots of the WFE RMS and V and H PSSn.
The dome seeing related data is sampled at 5Hz.
There are PSSn plots for both short exposure (SE) and long exposure (LE) PSSn.
SE PSSn have the same sampling than the dome seeing data (5Hz).
The LE PSSn values are derived from cummulative dome seeing images sampled at 1Hz (for the LE PSSn, the meaning of the x-axis is exposure time).

\paragraph{Wind loads}

The wind loads are detailed in part~\ref{wind-loads}.
Each section includes a snapshot of the vorticity for the whole Observatory,
two tables providing the mean, the standard deviation, the minimum and the
maximum of the magnitude of either the force or the moment exerted on the
elements of the telescope and time series plots of the force magnitude on the
different telescope elements.
Both, forces and moments are sampled at 20Hz.

The simulations take some time to converge to a stationary solution and this convergence time is a function of the initial conditions but strongly depends on the wind speed and the enclosure configuration.
The duration of each simulations have been set accordingly.
It is usually the case that the last 400s of the simulations are stationary and
the statistical quantity reported in the forces and moments tables have been derived from
the last 400s of the corresponding time series.
A mistake was made in one report of one C-ring element for 4 cases (Sec.~\ref{zen30az000_OS7}, Sec.~\ref{zen30az045_OS7}, Sec.~\ref{zen30az000_CD12} and Sec.~\ref{zen30az045_OS2}), the erroneous C-ring report have been corrected and the cases have been run for 400 more seconds. 

\paragraph{Heat transfer coefficients}
The HTC are reported in part~\ref{htc}.
Each section includes a table providing the mean, the standard deviation, the minimum and the
maximum of the HTC  on the elements of the telescope.
These quantities have been derived from the last 400s of the corresponding HTCs.
Like forces and moments, HTCs are sampled at 20Hz.

\paragraph{Excluded cases}
Three cases (zen00az090-CD17,  zen00az180-CD7,  zen60az000-OS2) have failed to run properly and have been removed from this report.
The configuration files for this cases have been regenerated and new simulation runs of these cases are currently in progress.
The results of these new runs will be added to the report once completed.

%%% Local Variables:
%%% mode: latex
%%% TeX-master: "gmto.cfd2021.mount"
%%% End:
