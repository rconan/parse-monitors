\section{Purpose}
\label{sec:purpose}

This document describes the results from the 2021/2022 CFD baseline simulations.

\section{Introduction}

Following the mount PDR\cite{MOUNT-PDR}, the CFD telescope model has been updated to take
into account the numerous design changes.
The updated CFD model is described in GMT-DOC--05211\cite{GMT.DOC.05211}.

There is a total of 60 CFD cases covering 3 telescope pointing angle (wrt. to zenith): 0$^o$, 30$^o$ and 60$^o$, 5 telescope azimuth angles wrt. 
the 30$^o$-NNE predominant wind direction at the GMT site: 0$^o$, 45$^o$, 90$^o$, 135$^o$ and 180$^o$, 5 wind speeds: 2m/s, 7m/s, 12m/s and 17m/s and 
3 enclosure configurations: open vents/stowed wind screen (OS), closed vents/deployed wind screen (CD) and closed vents/stowed wind screen (CS).
Table~\ref{tab:cases} identifies all the CFD cases.
\begin{table}
  \centering
  \begin{tabular}{c|cccc}\toprule
    Azimuth [degree] & \multicolumn{4}{c}{Wind speed [m/s]} \\\hline
     - & 2 & 7  & 12 & 17 \\\midrule
    \multicolumn{5}{c}{0$^o$ zenith}\\\hline
      0               & OS   & OS & CD    & CD        \\
     45               & OS   & OS & CD    & CD        \\
     90               & OS   & OS & CD    & CD        \\
    135               & OS   & OS & CD    & CD        \\
    180               & OS   & OS & CD    & CD        \\\hline
    \multicolumn{5}{c}{30$^o$ zenith}\\\hline
      0               & OS   & OS & CD    & CD        \\
     45               & OS   & OS & CD    & CD        \\
     90               & OS   & OS & CD    & CD        \\
    135               & OS   & OS & CD    & CD        \\
    180               & OS   & OS & CD    & CD        \\\hline
    \multicolumn{5}{c}{60$^o$ zenith}\\\hline
      0               & OS   & OS & CS    & CS        \\
     45               & OS   & OS & CS    & CS        \\
     90               & OS   & OS & CS    & CS        \\
    135               & OS   & OS & CS    & CS        \\
    180               & OS   & OS & CS    & CS        \\
    \bottomrule
  \end{tabular}
  \caption[CFD GMT enclosure configuration]{CFD GMT enclosure configuration
   (OS: open vents/stowed wind screen, 
   CD: closed vents/deployed wind screen, 
   CS: closed vents/stowed wind screen) for all the CFD cases.}
  \label{tab:cases}
\end{table}

The report is divided into 3 parts, each corresponding to a specific data products from the CFD simulations: dome seeing in part~\ref{dome-seeing}, wind loads in part~\ref{wind-loads} and heat transfer coefficients (HTC) in part~\ref{htc}.
All the parts are divided into 3 chapters, one chapter for each telescope zenith angle.
In each chapter, there are as many sections as CFD cases at that particular zenith angle.

\paragraph{Dome seeing}

Dome seeing results are given in part~\ref{dome-seeing}.
Each chapter starts with a table with the WFE RMS and V and H PSSn values for both the current (2022) and the former (2020) set of simulations.
Each section includes a snapshot of the gradient of the refraction index for the whole Observatory and time series plots of the WFE RMS and V and H PSSn.
The dome seeing related data is sampled at 5Hz.
There are PSSn plots for both short exposure (SE) and long exposure (LE) PSSn.
SE PSSn have the same sampling than the dome seeing data (5Hz).
The LE PSSn values are derived from cummulative dome seeing images sampled at 1Hz (for the LE PSSn, the meaning of the x-axis is exposure time).

\paragraph{Wind loads}

The wind loads are detailed in part~\ref{wind-loads}.
Each section includes
\begin{itemize}
  \item a snapshot of the vorticity for the whole Observatory,
\item two tables providing the mean, the standard deviation, the minimum and the
maximum of the magnitude of either the force or the moment exerted on the
elements of the telescope, 
\item time series plots of the force magnitude on the
different telescope elements 
\item snapshots of the pressure on M1 and M2 segments at the end of the simulation.
\item the standard deviation of the rigid body motions of M1 and M2 segments and of the segment piston, tip and tilt
\end{itemize}
Both, forces and moments are sampled at 20Hz.
The rigid body motions are derived using the CFD loads (forces and moments) processed into equivalent forces and moments 
that can then be applied to a Finite Element Model (FEM) of the telescope\cite{GMT.DOC.05506}.
Three FEMs have been used to compute the rigid body motions, one for each zenith angle \cite{20220611_1945_MT_mount_zen_00_m1HFN_FSM_,20220610_1023_MT_mount_zen_30_m1HFN_FSM_,20220621_1443_MT_mount_zen_60_m1HFN_FSM_}.
They all include the PDR mount FEM, the telescope pier,
 the M1 segments and the FSM M2 segments.
Segment piston, tip and tilt are computed by multiplying M1 and M2 rigid body motions
 with linear optical sensitivities derived with CEO\cite{CEO}.

The simulations take some time to converge to a stationary solution and this convergence time is a function of the initial conditions but strongly depends on the wind speed and the enclosure configuration.
The duration of each simulations have been set accordingly.
It is usually the case that the last 400s of the simulations are stationary and
the statistical quantity reported here have been derived from
the last 400s of the corresponding time series.

\paragraph{Heat transfer coefficients}
The HTC are reported in part~\ref{htc}.
Each section includes a table providing the mean, the standard deviation, the minimum and the
maximum of the HTC  on the elements of the telescope.
These quantities have been derived from the last 400s of the corresponding HTCs.
Like forces and moments, HTCs are sampled at 20Hz.

%%% Local Variables:
%%% mode: latex
%%% TeX-master: "gmto.cfd2021"
%%% End:
